This class will introduce students to the computational tools that are used to get things done in scientific research. Such tools include, but are not limited to, unix bash shell scripting, LaTeX/Beamer, virtual machines, git and github, tools for parallel computation, cloud services, and others. Brief treatments of special-purpose tools (like Mathematica for symbolic math) will conclude this part of the class. After this introduction, the course will involve an intensive introduction to the use of the Python language for scientific computation purposes, including a discussion of why Python dominates other choices like Matlab and Julia. The final third of the course will apply the tools in a practical application to a specific problem identified jointly between the instructor and the student. There is no required text; readings will be assigned in class. (The characteristic that distinguishes this class from alternatives is that this class will not teach specific algorithms nor frontier computational techniques; rather, it aims to expose students to a broad set of tools that they will use regularly thereafter).
 
